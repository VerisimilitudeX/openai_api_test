\documentclass{article}
\usepackage{amsmath}
\documentclass[8pt]{extarticle}
\usepackage[fontsize=8pt]{fontsize}
\usepackage{amssymb}

\title{CSE/STAT 416 - Comprehensive Review Notes}
\author{}
\date{}

\begin{document}

\maketitle

\section{Linear Regression Model}

\subsection{Fitting a regression model}
\begin{enumerate}
    \item Find the best line: $\text{RSS}(w_0, w_1) = \sum_{i=1}^N (y_i - [w_0 + w_1x_i])^2$ (quality metric)
    \begin{itemize}
        \item RSS (Residual Sum of Squares) measures the total squared difference between predicted and actual values
        \item Minimizing RSS is equivalent to finding the best-fitting line
    \end{itemize}
    
    \item Gradient Descent
    \begin{itemize}
        \item An iterative optimization algorithm used to find the minimum of a function
        \item Updates parameters in the opposite direction of the gradient of the loss function
        \item Learning rate determines the step size at each iteration
    \end{itemize}
    
    \item Regression model: $y_i = w_0 + w_1x_i + \epsilon_i$ 
    \begin{itemize}
        \item $w_0$: intercept (value of y when x = 0)
        \item $w_1$: slope (change in y per unit change in x)
        \item $\epsilon_i$: error term (captures noise and other unexplained factors)
    \end{itemize}
    
    \item Polynomial regression
    \begin{enumerate}
        \item Parameter/Coefficient: $w_j$ (scalar)
        \item $x[j] = j^{\text{th}}$ input (scalar)
        \item $h_j(x) = j^{\text{th}}$ feature (scalar)
        \item $x_i = $ input of $i^{\text{th}}$ data point (vector)
        \item $x_i[j] = j^{\text{th}}$ input of $i^{\text{th}}$ data point (scalar)
    \end{enumerate}
    
    \item General model: $y_i = w_0h_0(x_i) + w_1h_1(x_i) + \ldots + w_Dh_D(x_i) + \epsilon_i = \sum_{j=0}^D w_jh_j(x_i) + \epsilon_i$
    \begin{itemize}
        \item Allows for more complex relationships between inputs and outputs
        \item Can capture non-linear patterns in the data
    \end{itemize}
\end{enumerate}

\subsection{Assessing the model}
\begin{enumerate}
    \item Loss functions:
    \begin{itemize}
        \item Squared error: $(y_i - f(x_i))^2$
            \begin{itemize}
                \item Penalizes large errors more heavily
                \item Sensitive to outliers
            \end{itemize}
        \item Absolute error: $|y_i - f(x_i)|$
            \begin{itemize}
                \item Less sensitive to outliers compared to squared error
                \item May be harder to optimize due to non-differentiability at zero
            \end{itemize}
        \item Huber loss: Combines properties of squared error and absolute error
            \begin{itemize}
                \item Less sensitive to outliers than squared error
                \item Differentiable everywhere, unlike absolute error
            \end{itemize}
    \end{itemize}
    
    \item Training error: $\frac{1}{N} \sum_{i=1}^N L(y_i, f_{\hat{w}}(x_i))$
    \begin{itemize}
        \item Measures how well the model fits the training data
        \item RMSE (Root Mean Squared Error) = $\sqrt{\frac{1}{N} \sum_{i=1}^N (y_i - f_w(x_i))^2}$
            \begin{itemize}
                \item Provides error in the same units as the target variable
                \item Commonly used for regression problems
            \end{itemize}
        \item Training error vs. model complexity: typically decreases as complexity increases
    \end{itemize}
    
    \item Generalization (true) error
    \begin{itemize}
        \item Expected error on new, unseen data
        \item Estimated using hold-out test set or cross-validation
        \item Generalization error vs. model complexity: U-shaped curve (bias-variance tradeoff)
    \end{itemize}
    
    \item Test error: $\frac{1}{N_{\text{test}}} \sum_{i \in \text{test set}} L(y_i, f_{\hat{w}}(x_i))$ (fit using training data)
    \begin{itemize}
        \item Provides an unbiased estimate of the model's performance on new data
        \item Should only be used once model selection and hyperparameter tuning are complete
    \end{itemize}
    
    \item Sources of error
    \begin{enumerate}
        \item Noise (Irreducible Error, $\epsilon_i$)
            \begin{itemize}
                \item Inherent variability in the data that cannot be explained by the model
                \item Sets a lower bound on the achievable error
            \end{itemize}
        \item Bias
            \begin{itemize}
                \item Error due to overly simplistic assumptions in the learning algorithm
                \item High bias (underfitting): model is too simple to capture the underlying pattern
                \item Low bias: model can capture complex patterns but may overfit
            \end{itemize}
        \item Variance
            \begin{itemize}
                \item Error due to the model's sensitivity to small fluctuations in the training set
                \item High variance (overfitting): model captures noise in the training data
                \item Low variance: model is more stable but may underfit
            \end{itemize}
    \end{enumerate}
\end{enumerate}

\subsection{Regularization}

\subsubsection{Ridge regression (L2 regularization)}
\begin{enumerate}
    \item Adds penalty term to the loss function: $\lambda \sum_{j=1}^p w_j^2$
    \item Objective: minimize $\text{RSS}(w) + \lambda \sum_{j=1}^p w_j^2$
    \item Effect of $\lambda$:
    \begin{itemize}
        \item Large $\lambda$: High bias, low variance (strong regularization)
        \item Small $\lambda$: Low bias, high variance (weak regularization)
        \item $\lambda = 0$: Equivalent to standard least squares
    \end{itemize}
    \item Coefficient path: Coefficients converge to 0 as $\lambda$ increases but never become exactly 0
    \item Helps prevent overfitting by shrinking coefficients towards zero
\end{enumerate}

\subsubsection{Lasso Regression (L1 regularization)}
\begin{enumerate}
    \item Adds penalty term to the loss function: $\lambda \sum_{j=1}^p |w_j|$
    \item Objective: minimize $\text{RSS}(w) + \lambda \sum_{j=1}^p |w_j|$
    \item Coefficient path: Features become exactly 0 as $\lambda$ increases
    \item Feature selection property: Can produce sparse models by setting some coefficients to exactly zero
    \item Useful when dealing with high-dimensional data or when feature selection is desired
\end{enumerate}

\section{Classification}

\subsection{Evaluation Metrics}
\begin{enumerate}
    \item Accuracy = $\frac{\text{Number of correct predictions}}{\text{Total number of predictions}}$
    \begin{itemize}
        \item Simple and intuitive measure of model performance
        \item Can be misleading for imbalanced datasets
    \end{itemize}
    
    \item Error rate = $\frac{\text{Number of incorrect predictions}}{\text{Total number of predictions}} = 1 - \text{Accuracy}$
    
    \item Confusion Matrix
    \begin{itemize}
        \item A table that describes the performance of a classification model
        \item Rows represent actual classes, columns represent predicted classes
        \item Entries: True Positives (TP), False Positives (FP), True Negatives (TN), False Negatives (FN)
    \end{itemize}
    
    \item Precision = $\frac{\text{TP}}{\text{TP} + \text{FP}}$
    \begin{itemize}
        \item Measures the accuracy of positive predictions
        \item Important when the cost of false positives is high
    \end{itemize}
    
    \item Recall (Sensitivity) = $\frac{\text{TP}}{\text{TP} + \text{FN}}$
    \begin{itemize}
        \item Measures the proportion of actual positives correctly identified
        \item Important when the cost of false negatives is high
    \end{itemize}
    
    \item F1 Score = $2 * \frac{\text{Precision} * \text{Recall}}{\text{Precision} + \text{Recall}}$
    \begin{itemize}
        \item Harmonic mean of precision and recall
        \item Provides a single score that balances both precision and recall
    \end{itemize}
    
    \item ROC (Receiver Operating Characteristic) Curve
    \begin{itemize}
        \item Plots True Positive Rate vs. False Positive Rate at various threshold settings
        \item AUC (Area Under the Curve) summarizes the model's performance in a single number
    \end{itemize}
\end{enumerate}

\subsection{Logistic Regression}
\begin{enumerate}
    \item Model: $P(y=+1|x_i, w) = \text{sigmoid}(\text{Score}(x_i)) = \frac{1}{1 + e^{-w^T h(x)}}$
    \begin{itemize}
        \item Estimates the probability of an instance belonging to a particular class
        \item Output is always between 0 and 1
    \end{itemize}
    
    \item Score$(x_i) = w_0 h_0(x) + w_1 h_1(x) + ... + w_D h_D(x_i)$
    \begin{itemize}
        \item Linear combination of features
        \item Passed through the sigmoid function to get probabilities
    \end{itemize}
    
    \item Maximum Likelihood Estimation
    \begin{itemize}
        \item Objective: Maximize the likelihood of observing the data given the model parameters
        \item Equivalent to minimizing the negative log-likelihood
    \end{itemize}
    
    \item Gradient Ascent for optimization
    \begin{itemize}
        \item Used to find the parameters that maximize the likelihood
        \item Updates parameters in the direction of the gradient of the log-likelihood
    \end{itemize}
    
    \item Regularization for logistic regression
    \begin{itemize}
        \item L1 (Lasso) and L2 (Ridge) regularization can be applied to logistic regression
        \item Helps prevent overfitting, especially in high-dimensional spaces
    \end{itemize}
    
    \item Overfitting in logistic regression
    \begin{itemize}
        \item Can occur when the model is too complex relative to the amount of training data
        \item Signs include large coefficient values and overconfident predictions
    \end{itemize}
\end{enumerate}

\section{Decision Trees}
\begin{enumerate}
    \item Splitting criteria
    \begin{itemize}
        \item Gini impurity: Measures the probability of incorrectly classifying a randomly chosen element
        \item Entropy and Information Gain: Measures the reduction in uncertainty achieved by a split
    \end{itemize}
    
    \item Recursive splitting
    \begin{itemize}
        \item Process of creating child nodes by splitting on the best feature at each step
        \item Continues until a stopping condition is met
    \end{itemize}
    
    \item Stopping conditions
    \begin{itemize}
        \item Maximum depth reached
        \item Minimum number of samples in a leaf node
        \item No improvement in the splitting criterion
    \end{itemize}
    
    \item Pruning
    \begin{itemize}
        \item Process of removing branches that do not provide significant predictive power
        \item Helps prevent overfitting
    \end{itemize}
    
    \item Advantages of decision trees
    \begin{itemize}
        \item Easy to interpret and explain
        \item Can handle both numerical and categorical data
        \item Requires little data preparation
    \end{itemize}
    
    \item Disadvantages of decision trees
    \begin{itemize}
        \item Can create overly complex trees that do not generalize well
        \item Unstable: small changes in the data can result in a very different tree
        \item Biased toward features with more levels (in the case of categorical variables)
    \end{itemize}
\end{enumerate}

\section{Ensemble Methods}

\subsection{Bagging (Bootstrap Aggregating)}
\begin{enumerate}
    \item Create multiple subsets of the original dataset by sampling with replacement
    \item Train a separate model on each subset
    \item Combine predictions by voting (classification) or averaging (regression)
    \item Reduces variance and helps prevent overfitting
    \item Random Forests: An extension of bagging that also randomly selects a subset of features for each split
\end{enumerate}

\subsection{Boosting}
\begin{enumerate}
    \item AdaBoost (Adaptive Boosting)
    \begin{itemize}
        \item Iteratively trains weak learners, focusing on misclassified examples
        \item Assigns weights to training examples, increasing weights for misclassified instances
        \item Combines weak learners into a strong learner by weighted majority voting
    \end{itemize}
    
    \item Gradient Boosting
    \begin{itemize}
        \item Builds trees sequentially, with each tree correcting the errors of the previous ones
        \item Optimizes an arbitrary differentiable loss function
        \item Popular implementations include XGBoost, LightGBM, and CatBoost
    \end{itemize}
\end{enumerate}

\section{Clustering and Similarity}

\subsection{Distance Metrics}
\begin{enumerate}
    \item Euclidean distance: $d(x,y) = \sqrt{\sum_{i=1}^n (x_i - y_i)^2}$
    \begin{itemize}
        \item Straight-line distance between two points in Euclidean space
        \item Sensitive to the scale of the features
    \end{itemize}
    
    \item Manhattan distance: $d(x,y) = \sum_{i=1}^n |x_i - y_i|$
    \begin{itemize}
        \item Sum of the absolute differences of the coordinates
        \item Less sensitive to outliers compared to Euclidean distance
    \end{itemize}
    
    \item Cosine similarity: $\text{similarity} = \frac{x \cdot y}{\|x\| \|y\|}$
    \begin{itemize}
        \item Measures the cosine of the angle between two vectors
        \item Useful for high-dimensional data and when the magnitude of the vectors is not important
    \end{itemize}
    
    \item Jaccard similarity: $J(A,B) = \frac{|A \cap B|}{|A \cup B|}$
    \begin{itemize}
        \item Measures similarity between sets
        \item Useful for comparing categorical data or binary vectors
    \end{itemize}
\end{enumerate}

\subsection{K-Means Clustering}
\begin{enumerate}
    \item Algorithm steps
    \begin{itemize}
        \item Initialize K cluster centers randomly
        \item Assign each point to the nearest cluster center
        \item Update cluster centers to be the mean of assigned points
        \item Repeat steps 2-3 until convergence or maximum iterations reached
    \end{itemize}
    
    \item Initialization methods
    \begin{itemize}
        \item Random initialization: Choose K points randomly as initial centers
        \item K-means++: Choose initial centers probabilistically, favoring points far from existing centers
    \end{itemize}
    
    \item Choosing K (elbow method)
    \begin{itemize}
        \item Plot within-cluster sum of squares (WCSS) vs. K
        \item Look for "elbow" point where WCSS starts to level off
    \end{itemize}
    
    \item Limitations of K-means
    \begin{itemize}
        \item Sensitive to initial conditions
        \item Assumes spherical clusters
        \item May converge to local optima
        \item Requires specifying K in advance
    \end{itemize}
\end{enumerate}

\subsection{Hierarchical Clustering}
\begin{enumerate}
    \item Agglomerative vs. Divisive
    \begin{itemize}
        \item Agglomerative: Bottom-up approach, start with each point as a cluster and merge
        \item Divisive: Top-down approach, start with all points in one cluster and split
    \end{itemize}
    
    \item Linkage methods
    \begin{itemize}
        \item Single linkage: Minimum distance between clusters
        \item Complete linkage: Maximum distance between clusters
        \item Average linkage: Average distance between all pairs of points in two clusters
    \end{itemize}
    
    \item Dendrogram interpretation
    \begin{itemize}
        \item Tree-like diagram showing hierarchical relationship between clusters
        \item Height of merge represents dissimilarity between clusters
        \item Can cut dendrogram at different heights to obtain different numbers of clusters
    \end{itemize}
\end{enumerate}

\section{Dimensionality Reduction}

\subsection{Principal Component Analysis (PCA)}
\begin{enumerate}
    \item Covariance matrix
    \begin{itemize}
        \item Symmetric matrix representing pairwise covariances between features
        \item Eigendecomposition of covariance matrix yields principal components
    \end{itemize}
    
    \item Eigenvectors and eigenvalues
    \begin{itemize}
        \item Eigenvectors represent directions of maximum variance
        \item Eigenvalues represent amount of variance explained by each eigenvector
    \end{itemize}
    
    \item Choosing number of components
    \begin{itemize}
        \item Based on cumulative explained variance ratio
        \item Scree plot: Look for elbow point in explained variance vs. number of components
    \end{itemize}
    
    \item Relationship to SVD
    \begin{itemize}
        \item PCA can be computed using Singular Value Decomposition (SVD)
        \item SVD decomposes matrix X into U, Σ, and V^T
        \item Principal components are columns of V
    \end{itemize}
\end{enumerate}

\section{Recommender Systems}

\subsection{Collaborative Filtering}
\begin{enumerate}
    \item User-based collaborative filtering
    \begin{itemize}
        \item Find similar users based on rating patterns
        \item Predict ratings based on ratings of similar users
    \end{itemize}
    
    \item Item-based collaborative filtering
    \begin{itemize}
        \item Find similar items based on user ratings
        \item Predict ratings based on user's ratings of similar items
    \end{itemize}
\end{enumerate}

\subsection{Matrix Factorization}
\begin{enumerate}
    \item SVD (Singular Value Decomposition)
    \begin{itemize}
        \item Factorize user-item matrix into user and item latent factor matrices
        \item Can handle missing values through iterative methods
    \end{itemize}
    
    \item Alternating Least Squares (ALS)
    \begin{itemize}
        \item Iteratively solve for user and item factors while holding the other fixed
        \item Can be parallelized for large-scale problems
    \end{itemize}
\end{enumerate}

\subsection{Content-Based Filtering}
\begin{enumerate}
    \item Feature extraction
    \begin{itemize}
        \item Extract relevant features from item descriptions or metadata
        \item Can use techniques like TF-IDF for text data
    \end{itemize}
    
    \item Similarity measures
    \begin{itemize}
        \item Cosine similarity, Jaccard similarity, etc.
        \item Compare item features to user preferences or previously liked items
    \end{itemize}
\end{enumerate}

\section{Deep Learning}

\subsection{Neural Network Basics}
\begin{enumerate}
    \item Activation functions
    \begin{itemize}
        \item ReLU: max(0, x)
        \item Sigmoid: 1 / (1 + e^(-x))
        \item Tanh: (e^x - e^(-x)) / (e^x + e^(-x))
    \end{itemize}
    
    \item Backpropagation
    \begin{itemize}
        \item Efficient algorithm for computing gradients in neural networks
        \item Uses chain rule to propagate errors backwards through the network
    \end{itemize}
    
    \item Gradient descent variants
    \begin{itemize}
        \item Stochastic Gradient Descent (SGD): Update weights after each sample
        \item Mini-batch SGD: Update weights after processing a small batch of samples
        \item Adam: Adaptive learning rate method
        \item RMSprop: Adaptive learning rate method that addresses issues with AdaGrad
    \end{itemize}
\end{enumerate}

\subsection{Convolutional Neural Networks (CNNs)}
\begin{enumerate}
    \item Convolutional layers
    \begin{itemize}
        \item Apply filters to input to extract features
        \item Share weights across spatial dimensions
        \item Reduce number of parameters compared to fully connected layers
    \end{itemize}
    
    \item Pooling layers
    \begin{itemize}
        \item Reduce spatial dimensions of feature maps
        \item Max pooling: Take maximum value in pooling window
        \item Average pooling: Take average value in pooling window
    \end{itemize}
    
    \item Fully connected layers
    \begin{itemize}
        \item Usually used at the end of CNN for classification
        \item Connect all neurons from previous layer to all neurons in current layer
    \end{itemize}
\end{enumerate}

\subsection{Transfer Learning}
\begin{enumerate}
    \item Fine-tuning
    \begin{itemize}
        \item Start with pre-trained model on large dataset
        \item Replace and retrain final layers on new task
        \item Optionally fine-tune earlier layers with small learning rate
    \end{itemize}
    
    \item Feature extraction
    \begin{itemize}
        \item Use pre-trained model as fixed feature extractor
        \item Train new classifier on extracted features for new task
    \end{itemize}
\end{enumerate}

\subsection{Regularization Techniques}
\begin{enumerate}
    \item Dropout
    \begin{itemize}
        \item Randomly set a fraction of neurons to zero during training
        \item Prevents co-adaptation of neurons
        \item Acts as an ensemble method
    \end{itemize}
    
    \item Batch normalization
    \begin{itemize}
        \item Normalize activations of each layer
        \item Reduces internal covariate shift
        \item Allows higher learning rates and faster convergence
    \end{itemize}
    
    \item L1/L2 regularization
    \begin{itemize}
        \item Add penalty term to loss function based on weight magnitudes
        \item L1: Encourages sparsity
        \item L2: Encourages smaller weights overall
    \end{itemize}
    
    \item Early stopping
    \begin{itemize}
        \item Stop training when validation error starts to increase
        \item Prevents overfitting by avoiding unnecessary training iterations
    \end{itemize}
\end{enumerate}

\end{document}